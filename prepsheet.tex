\documentclass[17pt]{extarticle}


\usepackage{amsmath, amssymb, amsthm, amsfonts, mathrsfs}
\usepackage{times, flexisym, mdframed, xcolor}
\usepackage{ulem,multicol}
\usepackage{mathtools}
\usepackage{tikz}
\usepackage{hyperref}
\usepackage{graphicx}
\usepackage{fancyhdr}
\usepackage{tikz-cd}%   Margins
% \usepackage[left=1in,right=1in, top=2in, bottom=2in]{geometry}
\usepackage[paperwidth= 8in,paperheight=5in,left=.25in,right=.25in, top=.25in, bottom=.25in]{geometry}
%

\usepackage{draculatheme}
\mdfdefinestyle{darkAnswer}{%
  fontcolor=draculafg,
  backgroundcolor=draculabg,
  linecolor=draculafg,
  }
\mdfdefinestyle{darkQuesion}{%
  fontcolor=draculafg,
  backgroundcolor=draculabg,
  linecolor=draculafg,
  linewidth=1pt
  }
% \mdfdefinestyle{darkAnswer}{%
%    }
% \mdfdefinestyle{darkQuesion}{%
%   linewidth=1pt
%    }

% --------------------------------------------------------------
%                         New Commands
% --------------------------------------------------------------
\newcommand{\m}{\scalebox{0.5}[1.0]{$-$}}
\newcommand{\lrb}[1]{\left[#1\right]}
\newcommand{\lrp}[1]{\left(#1\right)}
\newcommand{\lrs}[1]{\left\{#1\right\}}
\newcommand{\lra}[1]{\left<#1\right>}
\newcommand{\gof}[1]{g\left(#1\right)}
\newcommand{\fof}[1]{f\left(#1\right)}
\newcommand{\dof}[1]{d\left(#1\right)}
\newcommand{\kof}[1]{k\left(#1\right)}
\newcommand{\mof}[1]{m\left(#1\right)}
\newcommand{\pof}[1]{\phi\left(#1\right)}
\newcommand{\om}[1]{m^{\ast}\left(#1\right)}
\newcommand{\measure}[1]{m\left(#1\right)}
\newcommand{\supmet}[1]{\left\|#1\right\|_{\infty}}
\newcommand{\hof}[1]{h\left(#1\right)}
\newcommand{\met}[3]{\rho_{#1}\lrp{#2,#3}}
\newcommand{\IR}{\mathscr{R}}
\newcommand{\rank}[1]{\text{rank}\left(#1\right)}
\newcommand{\sof}[1]{\sigma\left(#1\right)}
\newcommand{\Zof}[1]{Z\left(#1\right)}
\newcommand{\gofinv}[1]{g^{\m1}\left(#1\right)}
\newcommand{\fofinv}[1]{f^{\m1}\left(#1\right)}
\newcommand{\Zofinv}[1]{Z^{\m1}\left(#1\right)}
\newcommand{\pofinv}[1]{\ph{i=1}^{\m1}\left(#1\right)}
\newcommand{\nmod}[2]{#1\,\left(\text{mod}\,#2\right)}
\newcommand{\ngcd}[2]{\text{gcd}\left(#1\,,\,#2\right)}
\newcommand{\nlcm}[2]{\text{lcm}\left(#1\,,\,#2\right)}
\newcommand{\grp}[2]{\left(#1\,,\,#2\right)}
\newcommand{\GL}[2]{\mathrm{GL}_{#1}\lrp{#2}}
\newcommand{\SL}[2]{\mathrm{SL}_{#1}\lrp{#2}}
\newcommand{\Mn}[2]{M_{#1}\left(#2\right)}
\newcommand{\nord}[2]{\text{ord}_{#1}\left(#2\right)}
\newcommand{\ord}[1]{\text{ord}\left(#1\right)}
\newcommand{\dom}[1]{\text{dom}\left(#1\right)}
\newcommand{\ran}[1]{\text{ran}\left(#1\right)}
\newcommand{\degr}[1]{\text{deg}\left(#1\right)}
\newcommand{\krn}[1]{\text{ker}\left(#1\right)}
\newcommand{\intr}[1]{\text{int}\left(#1\right)}
\newcommand{\ball}[2]{B_{#1}\left(#2\right)}
\newcommand{\N}{\mathbb{N}}
\newcommand{\Z}{\mathbb{Z}}
\newcommand{\Q}{\mathbb{Q}}
\newcommand{\R}{\mathbb{R}}
\newcommand{\Opn}{\mathcal{O}}
\newcommand{\F}{\mathcal{F}}
\newcommand{\notsubset}{\not\subset}
% \newcommand{\right)}{\mathbb{R}^{+}}
\newcommand{\Dn}{\Delta^{n}}
\newcommand{\C}{\mathbb{C}}
\newcommand{\primeDec}{p_1^{\alpha_1}p_2^{\alpha_2}\cdots p_m^{\alpha_m}p_{m+1}^{\alpha_{m+1}}}
\newcommand{\abs}[1]{\left\vert#1\right\vert}
\newcommand{\norm}[1]{\left\vert\left\vert#1\right\vert\right\vert}
\newcommand{\Zm}[1]{\mathbb{Z}_{#1}}
\newcommand{\Zmx}[1]{\mathbb{Z}_{#1}^{\times}}
\newcommand{\Zp}{\mathbb{Z}_p}
\newcommand{\numb}[1]{\noindent{\bf #1)}}
\newcommand{\bigslant}[2]{{\raisebox{.2em}{$#1$}\left/\raisebox{-.2em}{$#2$}\right.}}
\newcommand{\inv}[1]{#1^{\m1}}
\newcommand{\totient}[1]{\varphi\left(#1\right)}
\newcommand{\vhalfpg}{\vspace{5in}}
\newcommand{\vthirdpg}{\vspace{3in}}
\newcommand{\vquartpg}{\vspace{2in}}
\newcommand{\Assoc}[1]{\item[Associativity:]{#1}}
\newcommand{\Invs}[1]{\item[Inverses:]{#1}}
\newcommand{\Clos}[1]{\item[Closure:]{#1}}
\newcommand{\Ident}[1]{\item[Identity:]{#1}}
\newcommand{\Abel}[1]{\item[Abelian:]{#1}}
\newcommand{\Tv}{\text{Tv}}
\newcommand{\V}{\text{V}}
\newcommand{\Ip}[1]{\text{Im}#1}
\newcommand{\LP}{\left(}
\newcommand{\RP}{\right)}
\newcommand{\LS}{\left\lbrace}
\newcommand{\RS}{\right\rbrace}
\newcommand{\LB}{\left[}
\newcommand{\RB}{\right]}
\newcommand{\MM}{\ \middle|\ }
\newcommand{\msr}[1]{m\left(#1\right)}
\newcommand{\dist}[1]{\text{d}\left(#1\right)}
\newcommand{\Diff}[3]{Diff_{#1}#2\left(#3\right)}
\newcommand{\Av}[3]{Av_{#1}#2\left(#3\right)}
\newcommand{\cball}[2]{\overline{B}_{#1}\left(#2\right)}
\newcommand{\opn}{\mathcal{O}}
\newcommand{\diam}{\operatorname{diam}}
\newcommand{\wind}[1]{n\lrp{\gamma;\ #1}}

\DeclarePairedDelimiter\ceil{\lceil}{\rceil}
\DeclarePairedDelimiter\floor{\lfloor}{\rfloor}
\newcommand{\twocase}[2]{\begin{enumerate}
                        \item[$ \implies$]{
                            #1
                            }
                        \bigskip
                        \item[$ \impliedby$]{
                          #2
                          }
                        \end{enumerate}}

% --------------------------------------------------------------
%                         Renew Commands
% --------------------------------------------------------------
\renewcommand{\det}[1]{\text{det}\left(#1\right)}
\renewcommand{\bar}[1]{\overline{#1}}
\renewcommand{\cos}[1]{\text{cos}\left(#1\right)}

\newcommand{\boxset}[2]{\begin{mdframed}[style=darkQuesion]
#1
\end{mdframed}
\newpage
\begin{mdframed}[style=darkQuesion]
  #1
    \end{mdframed}
\begin{mdframed}[style=darkAnswer]
  #2
    \end{mdframed}
    \newpage
}

\begin{document}

% --------------------------------------------------------------
%                         Start here
% --------------------------------------------------------------
% \pagestyle{fancy}
% \fancyhf{}
% \rhead{Math 5210 Homework 7}
% \lhead{ }
% \rfoot{Page \thepage}
\centering{{\fontsize{26pt}
{24pt}\selectfont{\underline{\smash{By Heart}}}}}\par
\newpage

\boxset{immersion at $p \in N$}
{A $C^{\infty} \operatorname{map} F: N \rightarrow M$ is said to be an immersion
at $p \in N$ if its differential $F_{*, p}: T_{p} N \rightarrow T_{F(p)} M$ is injective.}
\boxset{submersion at $p \in N$}
{A $C^{\infty} \operatorname{map} F: N \rightarrow M$ is said to be an submersion
at $p \in N$ if its differential $F_{*, p}: T_{p} N \rightarrow T_{F(p)} M$ is surjective.}
\boxset{immersion}
{We call $F$ an immersion
if it is an immersion at every $p \in N$}
\boxset{submersion}
{We call $F$ an submersion if it is a submersion
at every $p \in N$.}
\boxset{rank of a linear transformation, rank of a matrix}
{The rank of a linear transformation $L: V \rightarrow W$ between finite-dimensional vector spaces is the dimension of the image $L(V)$ as a subspace of $W$, while the rank of a matrix $A$ is the dimension of its column space. If $L$ is represented by a matrix $A$,  then the rank of $L$ is the rank of $A$, because $L(V)$ is the column space of $A$.}
\boxset{rank at a point $p$ in $N$}
{Let $F: N \rightarrow M$ be a smooth map of manifolds. Its rank at a point $p$ in $N$, denoted by $\operatorname{rk} F(p)$, is defined as the rank of the differential $F_{*, p}: T_{p} N \rightarrow T_{F(p)} M$. If $\lrp{U, x^1, \ldots, x^n}$ and $\lrp{V, y^1, \ldots, y^m}$ are coordinate neighborhoods about $p$ and $F(p)$ respectively, then

\begin{equation*}
	\operatorname{rk}F(p)  = \operatorname{rk}\begin{bmatrix}
		\frac{\partial F^i}{\partial x^j}(p)
	\end{bmatrix}
\end{equation*}

Where the matrix is the Jacobian.}
\boxset{critical point}
{A point $p$ in $N$ is a critical point of $F$ if the differential
\[F_{*, p}: T_{p} N \rightarrow T_{F(p)} M\]
fails to be surjective.}
\boxset{regular point}
{A point $p$ in $N$ is a regular point of $F$ if the differential $F_{*, p}$ is surjective; i.e., if and only if $F$ is a submersion at the point.}
\boxset{critical value, regular value}
{Let $F: N \rightarrow M$ be a smooth map of manifolds. A point in $M$ is a critical value if it is the image of a critical point; otherwise it is a regular value.
\[\ \]
Note: A point in $M$ is a critical value if even a single point in its preimage is a critical point. Thus, in order for a point in the image of $F$ to be a regular value, \textit{every} point in its preimage must be a regular point.}
\boxset{For a real-valued function $f: M \rightarrow \mathbb{R}$, a point $p$
in $M$ is a $\rule{1cm}{0.15mm}$ if and only if relative to some chart
$\left(U, x^{1}, \ldots, x^{n}\right)$ containing $p$, all the partial
derivatives satisfy
\[\rule{1cm}{0.15mm}\]}
{For a real-valued function $f: M \rightarrow \mathbb{R}$, a point $p$ in $M$ is a \textbf{critical point} if and only if relative to some chart $\left(U, x^{1}, \ldots, x^{n}\right)$ containing $p$, all the partial derivatives satisfy
\[\frac{\partial f}{\partial x^{j}}(p)=0, \quad j=1, \ldots, n\]}
\boxset{regular submanifold of dimension $k$}
{A subset $S$ of a manifold $N$ of dimension $n$ is a regular submanifold of dimension $k$ if for every $p \in S$ there is a coordinate neighborhood $(U, \phi)=\left(U, x^{1}, \ldots, x^{n}\right)$ of $p$ in the maximal atlas of $N$ such that $U \cap S$ is defined by the vanishing of $n-k$ of the coordinate functions.}
\boxset{adapted chart }
{An adapted chart is a chart fulfilling the conditions of the chart in the definition of a regular submanifold of dimension $k$, in particular: $(U, \phi)=\left(U, x^{1}, \ldots, x^{n}\right)$ of $p\in S$ is called an adapted chart if it is in the maximal atlas of $N$ and $U \cap S$ is defined by the vanishing of $n-k$ of the coordinate functions.}
\boxset{codimension}
{If $S$ is a regular submanifold of dimension $k$ in a manifold $N$ of dimension $n$, then $n-k$ is said to be the codimension of $S$ in $N$.}
\boxset{level set}
{A level set of a map $F: N \rightarrow M$ is a subset
\[F^{-1}(\{c\})=\{p \in N \mid F(p)=c\}\]
for some $c \in M$. The usual notation for a level set is $F^{-1}(c)$}
\boxset{level of a level set}
{The value $c \in M$ is called the level of the level set $F^{-1}(c)$.}
\boxset{regular level set}
{The inverse image $F^{-1}(c)$ of a regular value $c$ is called a regular level set.}
\boxset{Let $g: N \rightarrow \mathbb{R}$ be a $C^{\infty}$ function. A $\rule{1cm}{0.15mm}$ $g^{-1}(c)$ of level c of the function $g$ is the $\rule{1cm}{0.15mm}$ $f^{-1}(0)$ of the function $\rule{1cm}{0.15mm}$.}
{Let $g: N \rightarrow \mathbb{R}$ be a $C^{\infty}$ function. A regular level set $g^{-1}(c)$ of level c of the function $g$ is the regular zero set $f^{-1}(0)$ of the function $f=g-c$.}
\boxset{Regular level set theorem}
{Let $F: N \rightarrow M$ be a $C^{\infty}$ map of manifolds, with $\operatorname{dim} N=n$ and $\operatorname{dim} M=m$. Then a nonempty regular level set $F^{-1}(c)$, where $c \in M$, is a regular submanifold of $N$ of dimension equal to $n-m$.}
\boxset{A chart centered at a point}
{A chart about a point $p\in M$ is said to be centered about a point $p$ if $\phi(p)=0$. Given a maximal atlas, we can always find a chart centered about any point in a manifold.}
\boxset{Constant rank theorem}
{Let $N$ and $M$ be manifolds of dimensions $n$ and $m$ respectively. Suppose $f: N \rightarrow M$ has constant rank $k$ in a neighborhood of a point $p$ in $N$. Then there are charts $(U, \phi)$ centered at $p$ in $N$ and $(V, \psi)$ centered at $f(p)$ in $M$ such that for $\left(r^{1}, \ldots, r^{n}\right)$ in $\phi(U)$,
\[\left(\psi \circ f \circ \phi^{-1}\right)\left(r^{1}, \ldots, r^{n}\right)=\left(r^{1}, \ldots, r^{k}, 0, \ldots, 0\right)\]}
\boxset{Constant-rank level set theorem}
{Let $f: N \rightarrow M$ be a $C^{\infty}$ map of manifolds and $c \in M$. If $f$ has constant rank $k$ in a neighborhood of the level set $f^{-1}(c)$ in $N$, then $f^{-1}(c)$ is a regular submanifold of $N$ of codimension $k$.}
\boxset{
$f_{*, p} $ is $\rule{1cm}{0.15mm}$ $ \Longleftrightarrow n \leq m $ and $\operatorname{rk}\left[\partial f^{i} / \partial x^{j}(p)\right]=n$\[\ \]

$f_{*, p} $ is $\rule{1cm}{0.15mm}$ $ \Longleftrightarrow n \geq m $ and $ \operatorname{rk}\left[\partial f^{i} / \partial x^{j}(p)\right]=m $}
{$f_{*, p} $ is injective $ \Longleftrightarrow n \leq m $ and $\operatorname{rk}\left[\partial f^{i} / \partial x^{j}(p)\right]=n$\[\ \]
$f_{*, p} $ is surjective $ \Longleftrightarrow n \geq m $ and $ \operatorname{rk}\left[\partial f^{i} / \partial x^{j}(p)\right]=m $}
\boxset{Immersion theorem}
{Let $N$ and $M$ be manifolds of dimensions $n$ and $m$ respectively. Suppose $f: N \rightarrow M$ is an immersion at $p \in N$. Then there are charts $(U, \phi)$ centered at $p$ in $N$ and $(V, \psi)$ centered at $f(p)$ in $M$ such that in a neighborhood of $\phi(p)$,
\[\left(\psi \circ f \circ \phi^{-1}\right)\left(r^{1}, \ldots, r^{n}\right)=\left(r^{1}, \ldots, r^{n}, 0, \ldots, 0\right) .\]}
\boxset{Submersion theorem}
{Let $N$ and $M$ be manifolds of dimensions $n$ and $m$ respectively. Suppose $f: N \rightarrow M$ is a submersion at $p$ in $N$. Then there are charts $(U, \phi)$ centered at $p$ in $N$ and $(V, \psi)$ centered at $f(p)$ in $M$ such that in a neighborhood of $\phi(p)$,
\[\left(\psi \circ f \circ \phi^{-1}\right)\left(r^{1}, \ldots, r^{m}, r^{m+1}, \ldots, r^{n}\right)=\left(r^{1}, \ldots, r^{m}\right)\]}
\boxset{embedding}
{A $C^{\infty}$ map $f: N \rightarrow M$ is called an embedding if\[\ \]
(i) it is a one-to-one immersion and\[\ \]
(ii) the image $f(N)$ with the subspace topology is homeomorphic to $N$ under $f$. (The phrase "one-to-one" in this definition is redundant, since a homeomorphism is necessarily one-to-one.)}
\boxset{Suppose $f: N \rightarrow M$ is $C^{\infty}$ and the image of $f$ lies in a subset $S$ of $M$. If $S$ is a $\rule{1cm}{0.15mm}$ of $M$, then the $\rule{1cm}{0.15mm}$ is $C^{\infty}$.}
{Suppose $f: N \rightarrow M$ is $C^{\infty}$ and the image of $f$ lies in a subset $S$ of $M$. If $S$ is a regular submanifold of $M$, then the induced map $\tilde{f}: N \rightarrow S$ is $C^{\infty}$.}
\boxset{tangent bundle}
{The tangent bundle of $M$ is the union of all the tangent spaces of $M$ :
\[T M=\bigcup_{p \in M} T_{p} M\]}
\boxset{fiber}
{Given any map $\pi: E \rightarrow M$, we call the inverse image $\pi^{-1}(p):=\pi^{-1}(\{p\})$ of a point $p \in M$ the fiber at $p$. The fiber at $p$ is often written $E_{p}^{\prime \prime}$.}
\boxset{locally trivial of rank $r$}
{A surjective smooth map $\pi: E \rightarrow M$ of manifolds is said to be locally trivial of rank $r$ if\[\ \]
(i) each fiber $\pi^{-1}(p)$ has the structure of a vector space of dimension $r$;\[\ \]
(ii) for each $p \in M$, there are an open neighborhood $U$ of $p$ and a fiber-preserving diffeomorphism $\phi: \pi^{-1}(U) \rightarrow U \times \mathbb{R}^{r}$ such that for every $q \in U$ the restriction
\[\left.\phi\right|_{\pi^{-1}(q)}: \pi^{-1}(q) \rightarrow\{q\} \times \mathbb{R}^{r}\]
is a vector space isomorphism. Such an open set $U$ is called a trivializing open set for $E$, and $\phi$ is called a trivialization of $E$ over $U$.}
\boxset{$C^{\infty}$ vector bundle of rank $r$}
{A $C^{\infty}$ vector bundle of rank $r$ is a triple $(E, M, \pi)$ consisting of manifolds $E$ and $M$ and a surjective smooth map $\pi: E \rightarrow M$ that is locally trivial of rank $r$. The manifold $E$ is called the total space of the vector bundle and $M$ the base space. By abuse of language, we say that $E$ is a vector bundle over $M$. For any regular submanifold}
\boxset{section}
{A section of a vector bundle $\pi: E \rightarrow M$ is a map $s: M \rightarrow E$ such that $\pi \circ s=\mathbb{1}_{M}$, the identity map on $M$.}
\boxset{vector field $X$ on a manifold $M$}
{A vector field $X$ on a manifold $M$ is a function that assigns a tangent vector $X_{p} \in T_{p} M$ to each point $p \in M$. In terms of the tangent bundle, a vector field on $M$ is simply a section of the tangent bundle $\pi: T M \rightarrow M$ and the vector field is smooth if it is smooth as a map from $M$ to $T M$.}
\boxset{frame}
{A frame for a vector bundle $\pi: E \rightarrow M$ over an open set $U$ is a collection of sections $s_{1}, \ldots, s_{r}$ of $E$ over $U$ such that at each point $p \in U$, the elements $s_{1}(p), \ldots, s_{r}(p)$ form a basis for the fiber $E_{p}:=\pi^{-1}(p)$. A frame $s_{1}, \ldots, s_{r}$ is said to be smooth or $C^{\infty}$
if $s_{1}, \ldots, s_{r}$ are $C^{\infty}$ as sections of $E$ over $U$. A frame for the tangent bundle $T M \rightarrow M$ over an open set $U$ is simply called a frame on $U$.}
\boxset{support of a real-valued function}
{Recall that $\mathbb{R}^{\times}$ denotes the set of nonzero real numbers. The support of a real-valued function $f$ on a manifold $M$ is defined to be the closure in $M$ of the subset on which $f \neq 0$ :
\[\operatorname{supp} f=\operatorname{cl}_{M}\left(f^{-1}\left(\mathbb{R}^{\times}\right)\right)=\text {closure of }\{q \in M \mid f(q) \neq 0\} \text { in } M .{ }^{1}\]
Let $q$ be a point in $M$, and $U$ a neighborhood of $q$. }
\boxset{bump function }
{a bump function at $q$ supported in $U$ we mean any continuous nonnegative function $\rho$ on $M$ that is 1 in a neighborhood of $q$ with $\operatorname{supp} \rho \subset U$.}
\boxset{partition of unity}
{A $C^{\infty}$ partition of unity on a manifold is a collection of nonnegative $C^{\infty}$ functions $\left\{\rho_{\alpha}: M \rightarrow \mathbb{R}\right\} \alpha \in \mathrm{A}$ such that\[\ \]
(i) the collection of supports, $\left\{\operatorname{supp} \rho_{\alpha}\right\}_{\alpha \in A}$, is locally finite,\[\ \]
(ii) $\sum \rho_{\alpha}=1$.}
\boxset{subordinate to the open cover}
{Given an open cover $\left\{U_{\alpha}\right\}_{\alpha \in \mathrm{A}}$ of $M$, we say that a partition of unity $\left\{\rho_{\alpha}\right\}_{\alpha \in \mathrm{A}}$ is subordinate to the open cover $\left\{U_{\alpha}\right\}$ if $\operatorname{supp} \rho_{\alpha} \subset U_{\alpha}$ for every $\alpha \in \mathrm{A}$.}
\boxset{Existence of a $C^{\infty}$ partition of unity}
{Let $\left\{U_{\alpha}\right\}_{\alpha \in \mathrm{A}}$ be an open cover of a manifold $M$.\[\ \]
(i) There is a $C^{\infty}$ partition of unity $\left\{\varphi_{k}\right\}_{k=1}^{\infty}$ with every $\varphi_{k}$ having compact support such that for each $k$, supp $\varphi_{k} \subset U_{\alpha}$ for some $\alpha \in \mathrm{A}$.\[\ \]
(ii) If we do not require compact support, then there is a $C^{\infty}$ partition of unity $\left\{\rho_{\alpha}\right\}$ subordinate to $\left\{U_{\alpha}\right\}$.}
\boxset{ }
{}
\boxset{ }
{}
\boxset{ }
{}
\boxset{ }
{}
\boxset{ }
{}
\boxset{ }
{}
\boxset{ }
{}
\boxset{ }
{}
\boxset{ }
{}
\boxset{ }
{}
\boxset{ }
{}
\boxset{ }
{}
\boxset{ }
{}
\boxset{ }
{}
\boxset{ }
{}
\boxset{ }
{}
\boxset{ }
{}
\boxset{ }
{}
\boxset{ }
{}
\boxset{ }
{}
\boxset{ }
{}
\boxset{ }
{}
\boxset{ }
{}
\boxset{ }
{}
\boxset{ }
{}
\boxset{ }
{}
\boxset{ }
{}
\boxset{ }
{}
\boxset{ }
{}
\boxset{ }
{}
\boxset{ }
{}
\boxset{ }
{}
\boxset{ }
{}
\boxset{ }
{}
\boxset{ }
{}
\boxset{ }
{}
\boxset{ }
{}
\boxset{ }
{}
\boxset{ }
{}
\boxset{ }
{}
\boxset{ }
{}
\boxset{ }
{}
\boxset{ }
{}
\boxset{ }
{}
\boxset{ }
{}
\boxset{ }
{}
\boxset{ }
{}
\boxset{ }
{}
\boxset{ }
{}
\boxset{ }
{}
\boxset{ }
{}
\boxset{ }
{}
\boxset{ }
{}
\boxset{ }
{}
\boxset{ }
{}
\boxset{ }
{}
\boxset{ }
{}
\boxset{ }
{}
\boxset{ }
{}
\boxset{ }
{}
\boxset{ }
{}
\boxset{ }
{}
\boxset{ }
{}
\boxset{ }
{}
\boxset{ }
{}
\boxset{ }
{}
\boxset{ }
{}
\boxset{ }
{}
\boxset{ }
{}
\boxset{ }
{}
\boxset{ }
{}
\boxset{ }
{}
\boxset{ }
{}
\boxset{ }
{}
\boxset{ }
{}
\boxset{ }
{}
\boxset{ }
{}
\boxset{ }
{}
\boxset{ }
{}
\boxset{ }

\end{document}
