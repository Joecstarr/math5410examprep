\documentclass[17pt]{extarticle}


\usepackage{amsmath, amssymb, amsthm, amsfonts, mathrsfs}
\usepackage{times, flexisym, mdframed, xcolor}
\usepackage{ulem,multicol}
\usepackage{mathtools}
\usepackage{tikz}
\usepackage{hyperref}
\usepackage{graphicx}
\usepackage{fancyhdr}
\usepackage{tikz-cd}%   Margins
% \usepackage[left=1in,right=1in, top=2in, bottom=2in]{geometry}
\usepackage[paperwidth= 8in,paperheight=5in,left=.25in,right=.25in, top=.25in, bottom=.25in]{geometry}
%

\usepackage{draculatheme}
\mdfdefinestyle{darkAnswer}{%
  fontcolor=draculafg,
  backgroundcolor=draculabg,
  linecolor=draculafg,
  }
\mdfdefinestyle{darkQuesion}{%
  fontcolor=draculafg,
  backgroundcolor=draculabg,
  linecolor=draculafg,
  linewidth=1pt
  }
% \mdfdefinestyle{darkAnswer}{%
%    }
% \mdfdefinestyle{darkQuesion}{%
%   linewidth=1pt
%    }

% --------------------------------------------------------------
%                         New Commands
% --------------------------------------------------------------
\newcommand{\m}{\scalebox{0.5}[1.0]{$-$}}
\newcommand{\lrb}[1]{\left[#1\right]}
\newcommand{\lrp}[1]{\left(#1\right)}
\newcommand{\lrs}[1]{\left\{#1\right\}}
\newcommand{\lra}[1]{\left<#1\right>}
\newcommand{\gof}[1]{g\left(#1\right)}
\newcommand{\fof}[1]{f\left(#1\right)}
\newcommand{\dof}[1]{d\left(#1\right)}
\newcommand{\kof}[1]{k\left(#1\right)}
\newcommand{\mof}[1]{m\left(#1\right)}
\newcommand{\pof}[1]{\phi\left(#1\right)}
\newcommand{\om}[1]{m^{\ast}\left(#1\right)}
\newcommand{\measure}[1]{m\left(#1\right)}
\newcommand{\supmet}[1]{\left\|#1\right\|_{\infty}}
\newcommand{\hof}[1]{h\left(#1\right)}
\newcommand{\met}[3]{\rho_{#1}\lrp{#2,#3}}
\newcommand{\IR}{\mathscr{R}}
\newcommand{\rank}[1]{\text{rank}\left(#1\right)}
\newcommand{\sof}[1]{\sigma\left(#1\right)}
\newcommand{\Zof}[1]{Z\left(#1\right)}
\newcommand{\gofinv}[1]{g^{\m1}\left(#1\right)}
\newcommand{\fofinv}[1]{f^{\m1}\left(#1\right)}
\newcommand{\Zofinv}[1]{Z^{\m1}\left(#1\right)}
\newcommand{\pofinv}[1]{\ph{i=1}^{\m1}\left(#1\right)}
\newcommand{\nmod}[2]{#1\,\left(\text{mod}\,#2\right)}
\newcommand{\ngcd}[2]{\text{gcd}\left(#1\,,\,#2\right)}
\newcommand{\nlcm}[2]{\text{lcm}\left(#1\,,\,#2\right)}
\newcommand{\grp}[2]{\left(#1\,,\,#2\right)}
\newcommand{\GL}[2]{\mathrm{GL}_{#1}\lrp{#2}}
\newcommand{\SL}[2]{\mathrm{SL}_{#1}\lrp{#2}}
\newcommand{\Mn}[2]{M_{#1}\left(#2\right)}
\newcommand{\nord}[2]{\text{ord}_{#1}\left(#2\right)}
\newcommand{\ord}[1]{\text{ord}\left(#1\right)}
\newcommand{\dom}[1]{\text{dom}\left(#1\right)}
\newcommand{\ran}[1]{\text{ran}\left(#1\right)}
\newcommand{\degr}[1]{\text{deg}\left(#1\right)}
\newcommand{\krn}[1]{\text{ker}\left(#1\right)}
\newcommand{\intr}[1]{\text{int}\left(#1\right)}
\newcommand{\ball}[2]{B_{#1}\left(#2\right)}
\newcommand{\N}{\mathbb{N}}
\newcommand{\Z}{\mathbb{Z}}
\newcommand{\Q}{\mathbb{Q}}
\newcommand{\R}{\mathbb{R}}
\newcommand{\Opn}{\mathcal{O}}
\newcommand{\F}{\mathcal{F}}
\newcommand{\notsubset}{\not\subset}
% \newcommand{\right)}{\mathbb{R}^{+}}
\newcommand{\Dn}{\Delta^{n}}
\newcommand{\C}{\mathbb{C}}
\newcommand{\primeDec}{p_1^{\alpha_1}p_2^{\alpha_2}\cdots p_m^{\alpha_m}p_{m+1}^{\alpha_{m+1}}}
\newcommand{\abs}[1]{\left\vert#1\right\vert}
\newcommand{\norm}[1]{\left\vert\left\vert#1\right\vert\right\vert}
\newcommand{\Zm}[1]{\mathbb{Z}_{#1}}
\newcommand{\Zmx}[1]{\mathbb{Z}_{#1}^{\times}}
\newcommand{\Zp}{\mathbb{Z}_p}
\newcommand{\numb}[1]{\noindent{\bf #1)}}
\newcommand{\bigslant}[2]{{\raisebox{.2em}{$#1$}\left/\raisebox{-.2em}{$#2$}\right.}}
\newcommand{\inv}[1]{#1^{\m1}}
\newcommand{\totient}[1]{\varphi\left(#1\right)}
\newcommand{\vhalfpg}{\vspace{5in}}
\newcommand{\vthirdpg}{\vspace{3in}}
\newcommand{\vquartpg}{\vspace{2in}}
\newcommand{\Assoc}[1]{\item[Associativity:]{#1}}
\newcommand{\Invs}[1]{\item[Inverses:]{#1}}
\newcommand{\Clos}[1]{\item[Closure:]{#1}}
\newcommand{\Ident}[1]{\item[Identity:]{#1}}
\newcommand{\Abel}[1]{\item[Abelian:]{#1}}
\newcommand{\Tv}{\text{Tv}}
\newcommand{\V}{\text{V}}
\newcommand{\Ip}[1]{\text{Im}#1}
\newcommand{\LP}{\left(}
\newcommand{\RP}{\right)}
\newcommand{\LS}{\left\lbrace}
\newcommand{\RS}{\right\rbrace}
\newcommand{\LB}{\left[}
\newcommand{\RB}{\right]}
\newcommand{\MM}{\ \middle|\ }
\newcommand{\msr}[1]{m\left(#1\right)}
\newcommand{\dist}[1]{\text{d}\left(#1\right)}
\newcommand{\Diff}[3]{Diff_{#1}#2\left(#3\right)}
\newcommand{\Av}[3]{Av_{#1}#2\left(#3\right)}
\newcommand{\cball}[2]{\overline{B}_{#1}\left(#2\right)}
\newcommand{\opn}{\mathcal{O}}
\newcommand{\diam}{\operatorname{diam}}
\newcommand{\wind}[1]{n\lrp{\gamma;\ #1}}

\DeclarePairedDelimiter\ceil{\lceil}{\rceil}
\DeclarePairedDelimiter\floor{\lfloor}{\rfloor}
\newcommand{\twocase}[2]{\begin{enumerate}
                        \item[$ \implies$]{
                            #1
                            }
                        \bigskip
                        \item[$ \impliedby$]{
                          #2
                          }
                        \end{enumerate}}

% --------------------------------------------------------------
%                         Renew Commands
% --------------------------------------------------------------
\renewcommand{\det}[1]{\text{det}\left(#1\right)}
\renewcommand{\bar}[1]{\overline{#1}}
\renewcommand{\cos}[1]{\text{cos}\left(#1\right)}

\newcommand{\boxset}[2]{\begin{mdframed}[style=darkQuesion]
#1
\end{mdframed}
\newpage
\begin{mdframed}[style=darkQuesion]
  #1
    \end{mdframed}
\begin{mdframed}[style=darkAnswer]
  #2
    \end{mdframed}
    \newpage
}

\begin{document}

% --------------------------------------------------------------
%                         Start here
% --------------------------------------------------------------
% \pagestyle{fancy}
% \fancyhf{}
% \rhead{Math 5210 Homework 7}
% \lhead{ }
% \rfoot{Page \thepage}
\centering{{\fontsize{26pt}
{24pt}\selectfont{\underline{\smash{By Heart}}}}}\par
\newpage

\boxset{immersion at $p \in N$}
{A $C^{\infty} \operatorname{map} F: N \rightarrow M$ is said to be an immersion
at $p \in N$ if its differential $F_{*, p}: T_{p} N \rightarrow T_{F(p)} M$ is injective,}
\boxset{submersion at $p \in N$}
{A $C^{\infty} \operatorname{map} F: N \rightarrow M$ is said to be an submersion
at $p \in N$ if its differential $F_{*, p}: T_{p} N \rightarrow T_{F(p)} M$ is surjective,}
\boxset{immersion}
{We call $F$ an immersion
if it is an immersion at every $p \in N$}
\boxset{submersion}
{We call $F$ an submersion if it is a submersion
at every $p \in N$.}
\boxset{rank}
{The rank of a linear transformation $L: V \rightarrow W$ between finite-dimensional vector spaces is the dimension of the image $L(V)$ as a subspace of $W$, while the rank of a matrix $A$ is the dimension of its column space. If $L$ is represented by a matrix $A$}
\boxset{rank at a point $p$ in $N$}
{Now consider a smooth map $F: N \rightarrow M$ of manifolds. Its rank at a point $p$ in $N$, denoted by $\operatorname{rk} F(p)$, is defined as the rank of the differential $F_{*, p}: T_{p} N \rightarrow T_{F(p)} M$.}
\boxset{critical point}
{A point $p$ in $N$ is a critical point of $F$ if the differential
\[F_{*, p}: T_{p} N \rightarrow T_{F(p)} M\]
fails to be surjective.}
\boxset{regular point}
{It is a regular point of $F$ if the differential $F_{*, p}$ is surjective.}
\boxset{critical value or regular value}
{A point in $M$ is a critical value if it is the image of a critical point; otherwise it is a regular value }
\boxset{For a real-valued function $f: M \rightarrow \mathbb{R}$, a point $p$
in $M$ is a $\rule{1cm}{0.15mm}$ if and only if relative to some chart
$\left(U, x^{1}, \ldots, x^{n}\right)$ containing $p$, all the partial
derivatives satisfy
\[\rule{1cm}{0.15mm}\]}
{For a real-valued function $f: M \rightarrow \mathbb{R}$, a point $p$ in $M$ is a critical point if and only if relative to some chart $\left(U, x^{1}, \ldots, x^{n}\right)$ containing $p$, all the partial derivatives satisfy
\[\frac{\partial f}{\partial x^{j}}(p)=0, \quad j=1, \ldots, n\]}
\boxset{regular submanifold of dimension $k$}
{A subset $S$ of a manifold $N$ of dimension $n$ is a regular submanifold of dimension $k$ if for every $p \in S$ there is a coordinate neighborhood $(U, \phi)=\left(U, x^{1}, \ldots, x^{n}\right)$ of $p$ in the maximal atlas of $N$ such that $U \cap S$ is defined by the vanishing of $n-k$ of the coordinate functions.}
\boxset{adapted chart }
{For a chart of a regular submanifold defined by the vanishing of $n-k$ of the coordinate functions. We call such a chart $(U, \phi)$ in $N$ an adapted chart relative to $S$. }
\boxset{codimension}
{If $S$ is a regular submanifold of dimension $k$ in a manifold $N$ of dimension $n$, then $n-k$ is said to be the codimension of $S$ in $N$.}
\boxset{level set}
{A level set of a map $F: N \rightarrow M$ is a subset
\[F^{-1}(\{c\})=\{p \in N \mid F(p)=c\}\]
for some $c \in M$. The usual notation for a level set is $F^{-1}(c)$}
\boxset{level of a level set}
{The value $c \in M$ is called the level of the level set $F^{-1}(c)$.}
\boxset{regular level set}
{The inverse image $F^{-1}(c)$ of a regular value $c$ is called a regular level set.}
\boxset{Let $g: N \rightarrow \mathbb{R}$ be a $C^{\infty}$ function. A $\rule{1cm}{0.15mm}$ $g^{-1}(c)$ of level c of the function $g$ is the $\rule{1cm}{0.15mm}$ $f^{-1}(0)$ of the function $\rule{1cm}{0.15mm}$.}
{Let $g: N \rightarrow \mathbb{R}$ be a $C^{\infty}$ function. A regular level set $g^{-1}(c)$ of level c of the function $g$ is the regular zero set $f^{-1}(0)$ of the function $f=g-c$.}
\boxset{Regular level set theorem}
{Let $F: N \rightarrow M$ be a $C^{\infty}$ map of manifolds, with $\operatorname{dim} N=n$ and $\operatorname{dim} M=m$. Then a nonempty regular level set $F^{-1}(c)$, where $c \in M$, is a regular submanifold of $N$ of dimension equal to $n-m$.}
\boxset{Constant rank theorem}
{Let $N$ and $M$ be manifolds of dimensions $n$ and $m$ respectively. Suppose $f: N \rightarrow M$ has constant rank $k$ in a neighborhood of a point $p$ in $N$. Then there are charts $(U, \phi)$ centered at $p$ in $N$ and $(V, \psi)$ centered at $f(p)$ in $M$ such that for $\left(r^{1}, \ldots, r^{n}\right)$ in $\phi(U)$,
\[\left(\psi \circ f \circ \phi^{-1}\right)\left(r^{1}, \ldots, r^{n}\right)=\left(r^{1}, \ldots, r^{k}, 0, \ldots, 0\right)\]}
\boxset{Constant-rank level set theorem}
{Let $f: N \rightarrow M$ be a $C^{\infty}$ map of manifolds and $c \in M$. If $f$ has constant rank $k$ in a neighborhood of the level set $f^{-1}(c)$ in $N$, then $f^{-1}(c)$ is a regular submanifold of $N$ of codimension $k$.}
\boxset{ }
{}
\boxset{ }
{}
\boxset{ }
{}
\boxset{ }
{}
\boxset{ }
{}
\boxset{ }
{}
\boxset{ }
{}
\boxset{ }
{}
\boxset{ }
{}
\boxset{ }
{}
\boxset{ }
{}
\boxset{ }
{}
\boxset{ }
{}
\boxset{ }
{}
\boxset{ }
{}
\boxset{ }
{}
\boxset{ }
{}
\boxset{ }
{}
\boxset{ }
{}
\boxset{ }
{}
\boxset{ }
{}
\boxset{ }
{}
\boxset{ }
{}
\boxset{ }
{}
\boxset{ }
{}
\boxset{ }
{}
\boxset{ }
{}
\boxset{ }
{}
\boxset{ }
{}
\boxset{ }
{}
\boxset{ }
{}
\boxset{ }
{}
\boxset{ }
{}
\boxset{ }
{}
\boxset{ }
{}
\boxset{ }
{}
\boxset{ }
{}
\boxset{ }
{}
\boxset{ }
{}
\boxset{ }
{}
\boxset{ }
{}
\boxset{ }
{}
\boxset{ }
{}
\boxset{ }
{}
\boxset{ }
{}
\boxset{ }
{}
\boxset{ }
{}
\boxset{ }
{}
\boxset{ }
{}
\boxset{ }
{}
\boxset{ }
{}
\boxset{ }
{}
\boxset{ }
{}
\boxset{ }
{}
\boxset{ }
{}
\boxset{ }
{}
\boxset{ }
{}
\boxset{ }
{}
\boxset{ }
{}
\boxset{ }
{}
\boxset{ }
{}
\boxset{ }
{}
\boxset{ }
{}
\boxset{ }
{}
\boxset{ }
{}
\boxset{ }
{}
\boxset{ }
{}
\boxset{ }
{}
\boxset{ }
{}
\boxset{ }
{}
\boxset{ }
{}
\boxset{ }
{}
\boxset{ }
{}
\boxset{ }
{}
\boxset{ }
{}
\boxset{ }
{}
\boxset{ }
{}
\boxset{ }
{}
\boxset{ }
{}
\boxset{ }
{}
\boxset{ }
{}
\boxset{ }
{}
\boxset{ }
{}
\boxset{ }
{}
\boxset{ }
{}
\boxset{ }
{}
\boxset{ }
{}
\boxset{ }
{}
\boxset{ }
{}
\boxset{ }
{}
\boxset{ }
{}
\boxset{ }
{}
\boxset{ }
{}
\boxset{ }
{}
\boxset{ }
{}
\boxset{ }
{}
\boxset{ }

\end{document}
